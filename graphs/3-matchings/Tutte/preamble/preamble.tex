\documentclass[notheorems]{beamer}
%Information to be included in the title page:

\usefonttheme{serif} % нормальная тема

%%% Работа с русским языком
\usepackage{cmap}					% поиск в PDF
\usepackage{mathtext} 				% русские буквы в формулах
\usepackage[T2A]{fontenc}			% кодировка
\usepackage[utf8]{inputenc}			% кодировка исходного текста
\usepackage[russian]{babel}	% локализация и переносы
% \usepackage{indentfirst}            % красная строка в первом абзаце
\frenchspacing                      % равные пробелы между словами и предложениями

\uselanguage{russian}
\deftranslation[to=russian]{Theorem}{Теорема}
\deftranslation[to=russian]{Lemma}{Лемма}
\deftranslation[to=russian]{Definition}{Определение}

%%% Дополнительная работа с математикой
\usepackage{amsmath,amsfonts,amssymb,amsthm,mathtools} % пакеты AMS
\usepackage{icomma}                                    % "Умная" запятая

\setbeamertemplate{theorem}[ams style]

%%% Свои символы и команды
\usepackage{centernot} % центрированное зачеркивание символа
\usepackage{stmaryrd}  % некоторые спецсимволы

\renewcommand{\epsilon}{\ensuremath{\varepsilon}}
\renewcommand{\phi}{\ensuremath{\varphi}}
\renewcommand{\kappa}{\ensuremath{\varkappa}}
\renewcommand{\le}{\ensuremath{\leqslant}}
\renewcommand{\leq}{\ensuremath{\leqslant}}
\renewcommand{\ge}{\ensuremath{\geqslant}}
\renewcommand{\geq}{\ensuremath{\geqslant}}
\renewcommand{\emptyset}{\ensuremath{\varnothing}}

\DeclareMathOperator{\sgn}{sgn}
\DeclareMathOperator{\ke}{Ker}
\DeclareMathOperator{\im}{Im}
\DeclareMathOperator{\re}{Re}

\newcommand{\N}{\mathbb{N}}
\newcommand{\Z}{\mathbb{Z}}
\newcommand{\Q}{\mathbb{Q}}
\newcommand{\R}{\mathbb{R}}
\newcommand{\Cm}{\mathbb{C}}
\newcommand{\F}{\mathbb{F}}
\newcommand{\id}{\mathrm{id}}

\newcommand{\imp}[2]{
	(#1\,\,$\ra$\,\,#2)\,\,
}
\newcommand{\System}[1]{
	\left\{\begin{aligned}#1\end{aligned}\right.
}
\newcommand{\Root}[2]{
	\left\{\!\sqrt[#1]{#2}\right\}
}

\let\bs\backslash
\let\Lra\Leftrightarrow
\let\lra\leftrightarrow
\let\Ra\Rightarrow
\let\ra\rightarrow
\let\La\Leftarrow
\let\la\leftarrow
\let\emb\hookrightarrow

%%% DISPLAYSTYLE

%%% Перенос знаков в формулах (по Львовскому)
\newcommand{\hm}[1]{#1\nobreak\discretionary{}{\hbox{$\mathsurround=0pt #1$}}{}}

%%% Работа с картинками
\usepackage{graphicx}    % Для вставки рисунков
\setlength\fboxsep{4pt}  % Отступ рамки \fbox{} от рисунка
\setlength\fboxrule{1pt} % Толщина линий рамки \fbox{}
\usepackage{wrapfig}     % Обтекание рисунков текстом

%%% Работа с таблицами
\usepackage{array,tabularx,tabulary,booktabs} % Дополнительная работа с таблицами
\usepackage{longtable}                        % Длинные таблицы
\usepackage{multirow}                         % Слияние строк в таблице

%%% Оформление страницы
\usepackage{setspace}     % Интерлиньяж
% \usepackage{enumitem}     % Настройка окружений itemize и enumerate
\setlength\parindent{0pt}        % Устанавливает длину красной строки 15pt
\setlength{\parskip}{0.75em}      % Вертикальный интервал между абзацами
\linespread{1.0}
%\setcounter{secnumdepth}{0}      % Отключение нумерации разделов
%\setcounter{section}{-1}         % Нумерация секций с нуля
\usepackage{multicol}			  % Для текста в нескольких колонках
\usepackage{soulutf8}             % Модификаторы начертания

%%% Содержаниие
%\setlength{\cftsecnumwidth}{2.3em}
%\renewcommand{\cftsecdotsep}{1}
%\renewcommand{\cftsecpresnum}{\hfill}
%\renewcommand{\cftsecaftersnum}{\quad}

%%% Нумерация уравнений
\makeatletter
\def\eqref{\@ifstar\@eqref\@@eqref}
\def\@eqref#1{\textup{\tagform@{\ref*{#1}}}}
\def\@@eqref#1{\textup{\tagform@{\ref{#1}}}}
\makeatother                      % \eqref* без гиперссылки
\numberwithin{equation}{section}  % Нумерация вида (номер_секции).(номер_уравнения)
\mathtoolsset{showonlyrefs=false} % Номера только у формул с \eqref{} в тексте.

%%% Гиперссылки
\usepackage{hyperref}
\hypersetup{
	unicode=true,            % русские буквы в раздела PDF
	colorlinks=true,       	 % Цветные ссылки вместо ссылок в рамках
	linkcolor=black!15!blue, % Внутренние ссылки
	citecolor=green,         % Ссылки на библиографию
	filecolor=magenta,       % Ссылки на файлы
	urlcolor=NavyBlue,       % Ссылки на URL
}

%%% Графика
\usepackage{tikz}        % Графический пакет tikz
\usepackage{tikz-cd}     % Коммутативные диаграммы
\usepackage{tkz-euclide} % Геометрия
\usepackage{stackengine} % Многострочные тексты в картинках
\usetikzlibrary{angles, babel, quotes}

% Для кода
\usepackage[]{minted}


% Inkscape setup
\usepackage{import}
\usepackage{xifthen}
\usepackage{pdfpages}
\usepackage{transparent}

% SMILEYS !!!
\usepackage{wasysym}

\newcommand{\incfig}[1]{%
    \def\svgwidth{\columnwidth}
    \import{./figures/}{#1.pdf_tex}
}

\newcommand{\AuthorName}{Д. В. Карпов}
\newcommand{\SubjectName}{Теория графов}
\newcommand{\ChapterId}{0}
\newcommand{\ChapterName}{Теорема Татта}

\usepackage{fix-cm}

\newtranslation[to=Russian]{Definition}{Определение}

\title{\SubjectName. \ChapterName}
\author{\AuthorName}
\institute{Extended edition}
\date{2023}
\makeatletter
    \ifbeamer@countsect
      \newtheorem{theorem}{\translate{Theorem}}
    \else
      \newtheorem{theorem}{\translate{Theorem}}
    \fi
    \newtheorem{corollary}{\translate{Corollary}}
    \newtheorem{fact}{\translate{Fact}}
    \newtheorem{lemma}{\translate{Lemma}}
    \newtheorem{problem}{\translate{Problem}}
    \newtheorem{solution}{\translate{Solution}}

    \theoremstyle{definition}
    \newtheorem*{definition}{\translate{Definition}}
    \newtheorem{definitions}{\translate{Definitions}}

    \theoremstyle{example}
    \newtheorem{example}{\translate{Example}}
    \newtheorem{examples}{\translate{Examples}}
\makeatother
