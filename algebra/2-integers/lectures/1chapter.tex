\includepdf[pages=2-20]{2_integers}

\begin{frame}[t]
    \frametitle{\normalsize \bf Линейные диофантовы уравнения}
    \begin{itemize}
        \item $ax + by = c$ \textbf{(*)}, где  $a,b,c \in \Z$ -- константы, а $x, y \in \Z$ -- неизвестные.
        \item Если $c \nDiv \left( a,b \right)$, то, очевидно, (*) не имеет решений.
        \item Иначе, пусть $a = da', b = db', c = dc'$. Разделим (*) на $\left( a,b \right)$, и получим $a'x + b'y = c'$, где $\left( a',b' \right) = 1$.
        \item По Теореме 4, существует линейное представление НОДа: $a'x_0 + b'y_0 = 1 \overset{\cdot c'} \Leftrightarrow a'(x_0c') + b'(y_0c') = c'$. 
    \end{itemize}

\end{frame}

\begin{frame}[t]
    \setcounter{theorem}{9}
    \begin{theorem}[Решения диофанта]
        Решения (*) представляются в виде $x = x_0c' + tb'$, $y = y_0c' - ta'$, где $t \in \Z$.
    \end{theorem}
    
    \begin{proof}[Доказательство]
        Пусть мы нашли подходящую пару $\left( x_0, y_0 \right) $ для линейного представления $a'x + b'y = 1$. Тогда $\left( x_0c', y_0c' \right) $ -- пара, удовлетворяющая $a'x + b'y = c'$. Отсюда можно записать: $a'x + b'y = c' = a'(x_0c') + b'(y_0c') $ (подставили $\left( x_0c', y_0c \right)) $ Перегруппируем: $a'(x - x_0c') = b'(y_0c' - y)$ \textbf{(**)}. Так как $\left( a',b' \right) = 1$, а ЛЧ $\Div a'$, то остаётся единственный вариант --  $y_0c' - y \Div a' \Leftrightarrow y_0c' - y = a't \Leftrightarrow \fbox{$y = y_0c' - ta'$} (t \in \Z)$. Подставив в (**), получим: $a'(x - x_0c') = tb'a' \overset{a' \neq 0} \iff x - x_0c' = tb' \Leftrightarrow \fbox{$x = x_0c + tb'$}$.
    \end{proof}
    
\end{frame}

\includepdf[pages=23-30]{2_integers}

\begin{frame}[t]
    \frametitle{\bf \normalsize Теорема Эйлера}
    \small
    \setcounter{theorem}{14}
    \begin{theorem}[Эйлера]
        Пусть $m \in \N, a \in \Z, (a,m)=1$. Тогда $a^{\phi(m)} \equiv 1 \Mod m$
    \end{theorem}
    
    \begin{proof}[Доказательство]
        \begin{itemize}
            \item Пусть $r_1, \dots, r_{\phi(m)}$ -- ПрСВ $\Mod m$.
            \item По Теореме 14 $ar_1, \dots, ar_{\phi(m)}$ -- ПрСВ $\Mod m$.
            \item Перемножив элементы этих 2-х ПрСВ, мы получим, что они сравнимы$\Mod m$, потому что в каждой $\phi(m)$ чисел, и для каждого из первой найдётся какое-то сравнимое$\Mod m$ из второй. Пусть  $R = r_1 \dots r_{\phi(m)}$ 
            \item Тогда $a^{\phi(m)}R \equiv R \Mod m$. Так как $\forall i: (r_i, m) = 1 $, то на $R$ можно сократить: \fbox{$a^{\phi(m)} \equiv 1 \Mod m$}
        \end{itemize}
    \end{proof}
    \vspace{-0.7cm}
    \begin{corollary}[Малая теорема Ферма]
        Пусть $a \in \Z, p \in \mathbb{P}, (a, p) = 1$. Тогда $a^{p-1} \equiv 1 \Mod p$.
    \end{corollary}
\end{frame}

\includepdf[pages=32-34]{2_integers}

\begin{frame}[t]
    \frametitle{\bf \normalsize Сумма функции Эйлера по делителям числа}
    
    \setcounter{theorem}{16}
    \begin{theorem}[]
        Для любого $n \in \N: \displaystyle \sum_{d \mid n} \phi(d) = n$
    \end{theorem}

    \begin{proof}[Доказательство]
        \begin{itemize}
            \item Запишем в ряд дроби: $\frac{1}{n}, \frac{2}{n}, \dots, \frac{n}{n}$, приведём их к несократимому виду.
            \item Тогда множество знаменателей -- все делители числа $n$.
            \item Легко видеть, что для знаменателя $q$ всего существует $\phi(q)$ дробей (потому что дробь несократима).
            \item А всего выписано $n$ дробей. Значит, просуммировав по всем $q$ (то есть, по делителям $n$), получим:
                \begin{center}
                    $\displaystyle\sum_{d | n} \phi(d) = n$, \qedsymbol
                \end{center}
        \end{itemize}
    \end{proof}
\end{frame}

\includepdf[pages=36-39]{2_integers}

\begin{frame}[t]
    \frametitle{\bf \normalsize Китайская теорема об остатках}
    \small
    \setcounter{theorem}{18}
    \begin{theorem}
        Пусть $m_1, \dots, m_k$ -- попарно взаимно простые натуральные числа, $m = m_1 \dots m_k; a_1, \dots, a_k \in \Z$. Тогда существует единственное такое $a \in \left\{ 0, 1, \dots , m-1 \right \} $, удовлетворяющее системе сравнений.
    \end{theorem}

    \begin{proof}[Доказательство]
        Рассмотрим отображение $f: \Z_m \mapsto \Z_{m_1} \times \Z_{m_2} \times \dots \times \Z_{m_k}$, которое задано формулой $f(x) = \left( r_1, \dots, r_i, \dots, r_m \right)$. Здесь $r_i = a - qm_i, 0 \leq r_i < m_i$. Докажем, что отображение $f$ -- инъекция: пусть это не так, тогда для $x \neq y: f(x) = f(y)$, значит для любого $m_i: x \equiv r_i, y \equiv r_i \Rightarrow x-y \equiv 0 \Mod {m_i} $. Но тогда $x-y \Div m_i, i \in \left\{ 1, 2, \dots , k \right \} \Rightarrow x-y \Div m$. Но $x \neq y, 0 \leq x,y \leq m-1 \Rightarrow 0 < \left| x - y \right| \leq m-1$, противоречие. 

        Докажем, что это биекция: в $\Z_m$ всего $m$ элементов, как и в $\Z_{m_1} \times \Z_{m_2} \times \dots \times \Z_{m_k}$ (по правилу умножения). Значит, $f$ -- биекция, и каждой системе сравнений сопоставляется ровно одно решение.
    \end{proof}
\end{frame}

\includepdf[pages=42-44]{2_integers}

\begin{frame}[t]
    \frametitle{\normalsize \bf Формула обращения Мёбиуса}
    \setcounter{theorem}{19}

    \begin{theorem}[ФОМ]
        Пусть $f,g: \N \mapsto $, причём $g(n) = \sum\limits_{d \mid n} f(d)$ (сумма по делителям функции $f$). Тогда $f(n) = \sum\limits_{d \mid n} \left( \mu(\frac{n}{d}) g(d) \right) $.
    \end{theorem}

    \begin{proof}[Доказательство]
        \renewcommand{\qedsymbol}{}
        \begin{align*}
            \sum_{d | n} \left( \mu \left( \frac{n}{d} \right)  f(d) \right) = \sum_{d | n} \left( \mu\left(\frac{n}{d}\right) \cdot \sum_{d'| d}^{} g(d') \right) 
        \end{align*}
        Перегруппируем слагаемые: сначала будем суммировать по $d'$ ($d' \mid d \mid n \Rightarrow d' \mid n$), но тогда для каждого слагаемого этой суммы нужно закреплять $d: d' | d | n$. 
    \end{proof}
    
\end{frame}
\begin{frame}[t]
    \[
        \sum_{d | n} \left( \mu\left(\frac{n}{d}\right) \cdot \sum_{d'| d}^{} g(d') \right) = \sum_{d' | n}^{} \left( g(d') \cdot \sum_{d' | d | n}^{} \mu \left( \frac{n}{d} \right)  \right)
    \]
    Почти каждое слагаемое суммы равно 0, т. к. по Лемме 8 внутренняя сумма равна 0 всегда, кроме случая $d' = n$, но тогда подставим $d' = n$: \[
        \sum_{\substack{d' | n  \\ d' = n}}^{} \left( g(d') \cdot \sum_{d' | d | n}^{} \mu \left( \frac{n}{d} \right)  \right) = g(n) \sum_{n | d | n}^{} \mu \left( \frac{n}{d} \right) = g(n) \mu (1) = g(n), \qedsymbol
    \]
\end{frame}

\begin{frame}[t]
    \small
    \frametitle{\bf \normalsize Сумма по делителям функции Эйлера}
    \framesubtitle{Альтернативное доказательство}

    \begin{itemize}
        \item Предположим, что $\sum\limits_{d | n} \phi(d) = n$. 
        \item Мы уже знаем, что при $n = p_1^{\alpha_1} \dots p_k^{\alpha_k}$ значение функции Эйлера равно $\phi(n) = n\left( 1 - \frac{1}{p_1} \right) \left( 1 - \frac{1}{p_2} \right) \dots \left( 1 - \frac{1}{p_k} \right)$. 
        \item Значит, по ранее выведенной ФОМ достаточно показать, что $\phi(n) = \sum\limits_{d | n} \left( \mu \left( \frac{n}{d} \right) \cdot \sum\limits_{d' | d} \phi(d') \right) $.
        \item При $d = p_{i_1} \dots p_{i_t}$ мы имеем $\mu(d) = (-1)^t, \mu(1) = 1$, в остальных случаях  $\mu(d) = 0$. Поэтому, после технической работы слева равенство будет очевидно: \[
               n\left(1 - \sum_{1 \leq i \leq s} \frac{1}{p_i} + \sum_{1 \leq i_1 < i_2 \leq s}^{} \frac{1}{p_{i_1}p_{i_2}} - \dots \right)\overset{?} = 
 n\left( 1 - \frac{1}{p_1} \right) \dots \left( 1 - \frac{1}{p_k} \right)         \] 
    \end{itemize}
    
    \hfill \qedsymbol
\end{frame}
\includepdf[pages=46-52]{2_integers}

