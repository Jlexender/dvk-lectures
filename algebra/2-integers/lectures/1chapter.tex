\includepdf[pages=2-20]{2_integers}

\begin{frame}[t]
    \frametitle{\normalsize \bf Линейные диофантовы уравнения}
    \begin{itemize}
        \item $ax + by = c$ \textbf{(*)}, где  $a,b,c \in \Z$ -- константы, а $x, y \in \Z$ -- неизвестные.
        \item Если $c \nDiv \left( a,b \right)$, то, очевидно, (*) не имеет решений.
        \item Иначе, пусть $a = da', b = db', c = dc'$. Разделим (*) на $\left( a,b \right)$, и получим $a'x + b'y = c'$, где $\left( a,b \right) = 1$.
        \item По Теореме 4, существует линейное представление НОДа: $a'x_0 + b'y_0 = 1 \overset{\cdot c'} \Leftrightarrow a'(x_0c') + b'(y_0c') = c'$. 
    \end{itemize}

\end{frame}

\begin{frame}[t]
    \setcounter{theorem}{9}
    \begin{theorem}[Решения диофанта]
        Решения (*) представляются в виде $x = x_0c' + tb'$, $y = y_0c' - ta'$, где $t \in \Z$.
    \end{theorem}
    
    \begin{proof}[Доказательство]
        Пусть мы нашли подходящую пару $\left( x_0, y_0 \right) $ для линейного представления $a'x + b'y = 1$. Тогда $\left( x_0c', y_0c' \right) $ -- пара, удовлетворяющая $a'x + b'y = c'$. Отсюда можно записать: $a'x + b'y = c' = a'(x_0c') + b'(y_0c') $ (подставили $\left( x_0c', y_0c \right)) $ Перегруппируем: $a'(x - x_0c') = b'(y_0c' - y)$ \textbf{(**)}. Так как $\left( a',b' \right) = 1$, а ЛЧ $\Div a'$, то остаётся единственный вариант --  $y_0c' - y \Div a' \Leftrightarrow y_0c' - y = a't \Leftrightarrow \fbox{$y = y_0c' - ta'$} (t \in \Z)$. Подставив в (**), получим: $a'(x - x_0c') = tb'a' \overset{a' \neq 0} \iff x - x_0c' = tb' \Leftrightarrow \fbox{$x = x_0c + tb'$}$.
    \end{proof}
    
\end{frame}

\includepdf[pages=23-30]{2_integers}

\begin{frame}[t]
    \frametitle{\bf \normalsize Теорема Эйлера}
    \small
    \setcounter{theorem}{14}
    \begin{theorem}[Эйлера]
        Пусть $m \in \N, a \in \Z, (a,m)=1$. Тогда $a^{\phi(m)} \equiv 1 \Mod m$
    \end{theorem}
    
    \begin{proof}[Доказательство]
        \begin{itemize}
            \item Пусть $r_1, \dots, r_{\phi(m)}$ -- ПрСВ $\Mod m$.
            \item По Теореме 14 $ar_1, \dots, ar_{\phi(m)}$ -- ПрСВ $\Mod m$.
            \item Перемножив 2 эти ПрСВ, мы получим, что они сравнимы$\Mod m$, потому что в каждой $\phi(m)$ чисел, и для каждого из первой найдётся какое-то сравнимое$\Mod m$ из второй. Пусть  $R = r_1 \dots r_{\phi(m)}$ 
            \item Тогда $a^{\phi(m)}R \equiv R \Mod m$. Так как $\forall i: (r_i, m) = 1 $, то на $R$ можно сократить: \fbox{$a^{\phi(m)} \equiv 1 \Mod m$}
        \end{itemize}
    \end{proof}
    \vspace{-0.7cm}
    \begin{corollary}[Малая теорема Ферма]
        Пусть $a \in \Z, p \in \mathbb{P}, (a, p) = 1$. Тогда $a^{p-1} \equiv 1 \Mod p$.
    \end{corollary}
\end{frame}

\includepdf[pages=32-51]{2_integers}


